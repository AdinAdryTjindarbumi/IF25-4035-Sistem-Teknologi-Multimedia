\documentclass[11pt,a4paper]{article}
%%%%%%%%%%%%%%%%%%%%%%%%% Credit %%%%%%%%%%%%%%%%%%%%%%%%

% template ini dibuat oleh martin.manullang@if.itera.ac.id untuk dipergunakan oleh seluruh sivitas akademik itera.

%%%%%%%%%%%%%%%%%%%%%%%%% PACKAGE starts HERE %%%%%%%%%%%%%%%%%%%%%%%%
\usepackage{graphicx}
\usepackage{caption}
\usepackage{microtype}
\captionsetup[table]{name=Tabel}
\captionsetup[figure]{name=Gambar}
\usepackage{tabulary}
\usepackage{minted}
\usepackage{amsmath}
\usepackage{fancyhdr}
\usepackage{amssymb}
\usepackage{amsthm}
\usepackage{placeins}
\usepackage{amsfonts}
\usepackage{graphicx}
\usepackage[all]{xy}
\usepackage{tikz}
\usepackage{verbatim}
\usepackage[left=2cm,right=2cm,top=3cm,bottom=2.5cm]{geometry}
\usepackage{hyperref}
\hypersetup{
    colorlinks,
    linkcolor={red!50!black},
    citecolor={blue!50!black},
    urlcolor={blue!80!black}
}
\usepackage{caption}
\usepackage{subcaption}
\usepackage{multirow}
\usepackage{psfrag}
\usepackage[T1]{fontenc}
\usepackage[scaled]{beramono}
% Enable inserting code into the document
\usepackage{listings}
\usepackage{xcolor} 
% custom color & style for listing
\definecolor{codegreen}{rgb}{0,0.6,0}
\definecolor{codegray}{rgb}{0.5,0.5,0.5}
\definecolor{codepurple}{rgb}{0.58,0,0.82}
\definecolor{backcolour}{rgb}{0.95,0.95,0.92}
\definecolor{LightGray}{gray}{0.9}
\lstdefinestyle{mystyle}{
	backgroundcolor=\color{backcolour},   
	commentstyle=\color{green},
	keywordstyle=\color{codegreen},
	numberstyle=\tiny\color{codegray},
	stringstyle=\color{codepurple},
	basicstyle=\ttfamily\footnotesize,
	breakatwhitespace=false,         
	breaklines=true,                 
	captionpos=b,                    
	keepspaces=true,                 
	numbers=left,                    
	numbersep=5pt,                  
	showspaces=false,                
	showstringspaces=false,
	showtabs=false,                  
	tabsize=2
}
\lstset{style=mystyle}
\renewcommand{\lstlistingname}{Kode}
%%%%%%%%%%%%%%%%%%%%%%%%% PACKAGE ends HERE %%%%%%%%%%%%%%%%%%%%%%%%


%%%%%%%%%%%%%%%%%%%%%%%%% Data Diri %%%%%%%%%%%%%%%%%%%%%%%%
\newcommand{\student}{\textbf{Adin Adry Tjindarbumi (122140024)}}
\newcommand{\course}{\textbf{Sistem Teknologi Multimedia (IF25-40305)}}
\newcommand{\assignment}{\textbf{Worksheet 1: Setup Python Environment untuk Multimedia}}

%%%%%%%%%%%%%%%%%%% using theorem style %%%%%%%%%%%%%%%%%%%%
\newtheorem{thm}{Theorem}
\newtheorem{lem}[thm]{Lemma}
\newtheorem{defn}[thm]{Definition}
\newtheorem{exa}[thm]{Example}
\newtheorem{rem}[thm]{Remark}
\newtheorem{coro}[thm]{Corollary}
\newtheorem{quest}{Question}[section]
%%%%%%%%%%%%%%%%%%%%%%%%%%%%%%%%%%%%%%%%
\usepackage{lipsum}%% a garbage package you don't need except to create examples.
\usepackage{fancyhdr}
\pagestyle{fancy}
\lhead{Adin Adry Tjindarbumi (122140024)}
\rhead{ \thepage}
\cfoot{\textbf{Worksheet 1: Setup Python Environment untuk Multimedia}}
\renewcommand{\headrulewidth}{0.4pt}
\renewcommand{\footrulewidth}{0.4pt}

%%%%%%%%%%%%%%  Shortcut for usual set of numbers  %%%%%%%%%%%

\newcommand{\N}{\mathbb{N}}
\newcommand{\Z}{\mathbb{Z}}
\newcommand{\Q}{\mathbb{Q}}
\newcommand{\R}{\mathbb{R}}
\newcommand{\C}{\mathbb{C}}
\setlength\headheight{14pt}

%%%%%%%%%%%%%%%%%%%%%%%%%%%%%%%%%%%%%%%%%%%%%%%%%%%%%%%555
\begin{document}
\thispagestyle{empty}
\begin{center}
	\includegraphics[scale = 0.15]{Figure/ifitera-header.png}
	\vspace{0.1cm}
\end{center}
\noindent
\rule{17cm}{0.2cm}\\[0.3cm]
Nama: \student \hfill Tugas Ke: \assignment\\[0.1cm]
Mata Kuliah: \course \hfill Tanggal: \today\\
\rule{17cm}{0.05cm}
\vspace{0.1cm}



%%%%%%%%%%%%%%%%%%%%%%%%%%%%%%%%%%%%%%%%%%%%% BODY DOCUMENT %%%%%%%%%%%%%%%%%%%%%%%%%%%%%%%%%%%%%%%%%%%%%
\section{Tujuan Pembelajaran}
Setelah menyelesaikan worksheet ini, mahasiswa diharapkan mampu:
\begin{itemize}
    \item Memahami pentingnya manajemen environment Python untuk pengembangan multimedia
    \item Menginstall dan mengkonfigurasi Python environment menggunakan conda, venv, atau uv
    \item Menginstall library-library Python yang diperlukan untuk multimedia processing
    \item Memverifikasi instalasi dengan mengimpor dan menguji library multimedia
    \item Mendokumentasikan proses konfigurasi dan hasil pengujian dalam format \LaTeX
\end{itemize}

\section{Latar Belakang}
Python telah menjadi bahasa pemrograman yang sangat populer untuk multimedia processing karena memiliki ekosistem library yang sangat kaya. Namun, untuk dapat bekerja dengan multimedia secara efektif, kita perlu mengatur environment Python dengan benar dan menginstall library-library yang tepat.

Manajemen environment Python sangat penting untuk:
\begin{itemize}
    \item Menghindari konflik antar library (dependency conflict)
    \item Memastikan reproducibility dari project
    \item Memudahkan kolaborasi antar developer
    \item Memisahkan project yang berbeda dengan requirement yang berbeda
\end{itemize}

\section{Instruksi Tugas}

\subsection{Persiapan}
\textbf{Sebelum memulai, pastikan Anda telah:}
\begin{itemize}
    \item Menginstall Python 3.8 atau lebih baru di sistem Anda
    \item Memilih salah satu tool manajemen environment: \textbf{conda}, \textbf{venv}, atau \textbf{uv}
    \item Membuka terminal/command prompt
    \item Menyiapkan dokumen \LaTeX\ ini untuk dokumentasi
\end{itemize}

\subsection{Bagian 1: Membuat Environment Python}
Pilih \textbf{SALAH SATU} dari tiga opsi berikut dan ikuti langkah-langkahnya:

\subsubsection{Opsi 1: Menggunakan Conda (Direkomendasikan untuk pemula)}
Jalankan perintah berikut di terminal:

\begin{lstlisting}[language=bash, caption=Membuat environment dengan Conda]
# Membuat environment baru dengan nama 'multimedia'
conda create -n multimedia python=3.11

# Mengaktifkan environment
conda activate multimedia

# Verifikasi environment aktif
conda info --envs
\end{lstlisting}

\subsubsection{Opsi 2: Menggunakan venv (Built-in Python)}
\begin{lstlisting}[language=bash, caption=Membuat environment dengan venv]
# Membuat environment baru
python3 -m venv multimedia-env

# Mengaktifkan environment (Linux/Mac)
source multimedia-env/bin/activate

# Mengaktifkan environment (Windows)
# multimedia-env\Scripts\activate

# Verifikasi environment aktif
which python
\end{lstlisting}

\subsubsection{Opsi 3: Menggunakan uv (Modern dan cepat)}
\begin{lstlisting}[language=bash, caption=Membuat environment dengan uv]
# Install uv terlebih dahulu jika belum ada
# pip install uv

# Membuat environment baru
uv venv multimedia-uv

# Mengaktifkan environment (Linux/Mac)
source multimedia-uv/bin/activate

# Mengaktifkan environment (Windows)
# multimedia-uv\Scripts\activate

# Verifikasi environment aktif
which python
\end{lstlisting}

\textbf{Dokumentasikan di sini:}
\begin{itemize}
    \item Tool manajemen environment yang Anda pilih: \textbf{[UV]}
    \begin{figure}[H]
        \centering
        \includegraphics[width=0.9\textwidth]{image/image1.png}
        \caption{Perintah aktivasi environment uv}
        \label{fig:audio-dl}
    \end{figure}
\end{itemize}

\subsection{Bagian 2: Instalasi Library Multimedia}
Setelah environment aktif, install library-library berikut:

\subsubsection{Library Audio Processing}
\begin{lstlisting}[language=bash, caption=Instalasi library audio]
# Untuk conda:
conda install -c conda-forge librosa soundfile scipy

# Untuk pip (venv/uv):
pip install librosa soundfile scipy
\end{lstlisting}

\subsubsection{Library Image Processing}
\begin{lstlisting}[language=bash, caption=Instalasi library image]
# Untuk conda:
conda install -c conda-forge opencv pillow scikit-image matplotlib

# Untuk pip (venv/uv):
pip install opencv-python pillow scikit-image matplotlib
\end{lstlisting}

\subsubsection{Library Video Processing}
\begin{lstlisting}[language=bash, caption=Instalasi library video]
# Untuk conda:
conda install -c conda-forge ffmpeg
pip install moviepy

# Untuk pip (venv/uv):
pip install moviepy
\end{lstlisting}

\subsubsection{Library General Purpose}
\begin{lstlisting}[language=bash, caption=Instalasi library umum]
# Untuk conda:
conda install numpy pandas jupyter

# Untuk pip (venv/uv):
pip install numpy pandas jupyter
\end{lstlisting}

\clearpage
\textbf{Dokumentasikan di sini:}

\begin{itemize}
    \item Perintah instalasi yang Anda gunakan
    \begin{lstlisting}
pip install librosa soundfile scipy
pip install opencv-python pillow scikit-image matplotlib
pip install moviepy
pip install numpy pandas jupyter\end{lstlisting}
    \item Screenshot proses instalasi atau output sukses  
        \begin{figure}[H]
            \centering
            \includegraphics[width=0.9\textwidth]{image/audio-dl.png}
            \caption{Perintah instalasi library audio}
            \label{fig:audio-dl}
        \end{figure}

        \begin{figure}[H]
            \centering
            \includegraphics[width=0.9\textwidth]{image/image-dl.png}
            \caption{Perintah instalasi library image}
            \label{fig:image-dl}
        \end{figure}

        \begin{figure}[H]
            \centering
            \includegraphics[width=0.9\textwidth]{image/video-dl.png}
            \caption{Perintah instalasi library video}
            \label{fig:video-dl}
        \end{figure}
    
    \clearpage
    \item Daftar library yang berhasil diinstall dengan versinya
        \begin{figure}[H]
            \centering
            \includegraphics[width=0.5\textwidth]{image/lib1.png}
            \includegraphics[width=0.5\textwidth]{image/lib2.png}
            \caption{Daftar library yang berhasil diinstall}
            \label{fig:video-dl}
        \end{figure}

\end{itemize}

\subsection{Bagian 3: Verifikasi Instalasi}
Buat file Python sederhana untuk menguji semua library yang telah diinstall:


\textbf{Jalankan script dan dokumentasikan hasilnya:}

\subsection{Bagian 4: Simple Test dengan Sample Code}
Buat dan jalankan contoh sederhana untuk setiap kategori multimedia:

\subsubsection{Test Audio Processing}
\begin{lstlisting}[language=Python, caption=Test audio processing sederhana]
import numpy as np
import matplotlib.pyplot as plt

# Generate simple sine wave
duration = 2  # seconds
sample_rate = 44100
frequency = 440  # A4 note

t = np.linspace(0, duration, int(sample_rate * duration))
audio_signal = np.sin(2 * np.pi * frequency * t)

# Plot waveform
plt.figure(figsize=(10, 4))
plt.plot(t[:1000], audio_signal[:1000])  # Plot first 1000 samples
plt.title('Sine Wave (440 Hz)')
plt.xlabel('Time (s)')
plt.ylabel('Amplitude')
plt.grid(True)
plt.savefig('sine_wave_test.png', dpi=150, bbox_inches='tight')
plt.show()

print(f"Generated {duration}s sine wave at {frequency}Hz")
print(f"Sample rate: {sample_rate}Hz")
print(f"Total samples: {len(audio_signal)}")
\end{lstlisting}

\subsubsection{Test Image Processing}
\begin{lstlisting}[language=Python, caption=Test image processing sederhana]
import numpy as np
import matplotlib.pyplot as plt
from PIL import Image

# Create a simple test image
width, height = 400, 300
image = np.zeros((height, width, 3), dtype=np.uint8)

# Add some patterns
image[:, :width//3, 0] = 255  # Red section
image[:, width//3:2*width//3, 1] = 255  # Green section
image[:, 2*width//3:, 2] = 255  # Blue section

# Add a white circle in the center
center_x, center_y = width//2, height//2
radius = 50
Y, X = np.ogrid[:height, :width]
mask = (X - center_x)**2 + (Y - center_y)**2 <= radius**2
image[mask] = [255, 255, 255]

# Display and save
plt.figure(figsize=(8, 6))
plt.imshow(image)
plt.title('Test Image with RGB Stripes and White Circle')
plt.axis('off')
plt.savefig('test_image.png', dpi=150, bbox_inches='tight')
plt.show()

print(f"Created test image: {width}x{height} pixels")
print(f"Image shape: {image.shape}")
print(f"Image dtype: {image.dtype}")
\end{lstlisting}

\textbf{Dokumentasikan hasil eksekusi:}
\begin{itemize}
    \item Screenshot output dari kedua script di atas
        \begin{figure}[H]
            \centering
            \includegraphics[width=0.9\textwidth]{image/ss_test_audio.png}
            \caption{Output test audio processing}
            \label{fig:test-audio}
        \end{figure}
        \begin{figure}[H]
            \centering
            \includegraphics[width=0.9\textwidth]{image/ss_test_image.png}
            \caption{Output test image processing}
            \label{fig:test-audio}
        \end{figure}
    \item Gambar yang dihasilkan (sine\_wave\_test.png dan test\_image.png)
        \begin{figure}[H]
            \centering
            \includegraphics[width=0.9\textwidth]{image/sine_wave_test.png}
            \caption{Output gambar sine wave dari test audio processing}
            \label{fig:test-audio}
        \end{figure}
        \begin{figure}[H]
            \centering
            \includegraphics[width=0.9\textwidth]{image/test_image.png}
            \caption{Output gambar test image processing}
        \end{figure}
    \item Tidak ada error message 
\end{itemize}

\section{Bagian Laporan}

\subsection{Output Verifikasi Instalasi}
\textbf{Copy-paste output lengkap dari script \texttt{test\_multimedia.py} di sini:}

\begin{lstlisting}[caption=Output verifikasi instalasi]
#OUTPUT test_audio.py
PS D:\tugas1\multimedia-uv-adin> & C:/Users/ASUS/AppData/Local/Programs/Python/Python313/python.exe d:/tugas1/multimedia-uv-adin/test_audio.py
Generated 2s sine wave at 440Hz
Sample rate: 44100Hz
Total samples: 88200

# OUTPUT test_image.py
PS D:\tugas1\multimedia-uv-adin> & C:/Users/ASUS/AppData/Local/Programs/Python/Python313/python.exe d:/tugas1/multimedia-uv-adin/test_image.py
Created test image: 400x300 pixels
Image shape: (300, 400, 3)
Image dtype: uint8
\end{lstlisting}

\subsection{Screenshot Hasil Test}
\textbf{Sisipkan screenshot atau gambar hasil dari:}
\begin{itemize}
    \item Terminal/command prompt yang menunjukkan environment aktif
    \begin{figure}[H]
        \centering
        \includegraphics[width=0.9\textwidth]{image/activate.png}
        \caption{Environment uv aktif}
    \end{figure}

    \item Output dari script test audio (sine wave plot)
    \begin{figure}[H]
        \centering
        \includegraphics[width=0.9\textwidth]{image/figure_audio.png}
        \caption{Output test audio processing}
    \end{figure}

    \newpage
    \item Output dari script test image (RGB stripes dengan circle)
    \begin{figure}[H]
        \centering
        \includegraphics[width=0.9\textwidth]{image/figure_image.png}
        \caption{Output test image processing}
    \end{figure}
\end{itemize}

\subsection{Analisis dan Refleksi}
\textbf{Jawab pertanyaan berikut:}

\begin{enumerate}
    \item \textbf{Mengapa penting menggunakan environment terpisah untuk project multimedia?}
    
    \textit{Karena tiap project multimedia biasanya membutuhkan library yang berbeda-beda, sehingga dengan menggunakan environment yang terpisah, kita dapat meminimalisir potensi terjadinya konflik library yang bertabrakan.}
    
    \item \textbf{Apa perbedaan utama antara conda, venv, dan uv? Mengapa Anda memilih tool yang Anda gunakan?}
    
    \textit{Perbedaan yang mencolok antara conda, venv, dan uv menurut saya adalah conda lebih lengkap karena memiliki package manager tetapi ukuran filenya lebih besar, sedangkan venv lebih ringan dan merupakan tool bawaan python tetapi tidak memiliki package manager. Sedangkan uv lebih modern dan cepat dibanding dengan venv, serta memiliki fitur-fitur tambahan seperti manajemen dependency yang lebih baik. Alasan utmama saya memilih uv adalah karena ukurannya yang ringan dan sangat cepat dibandingkan conda dan venv.}
    
    \item \textbf{Library mana yang paling sulit diinstall dan mengapa?}
    
    \textit{Menurut saya, tidak ada library yang sulit untuk diinstall karena saat penginstalan tidak ada error yang muncul dan semua berjalan dengan lancar.}
    
    \item \textbf{Bagaimana cara mengatasi masalah dependency conflict jika terjadi?}
    
    \textit{Berdasarkan yang saya ketahui, caranya adalah dengan membuat environment baru dan apabila perlu menginstall kembali library yang diperlukan satu persatu.}
    
    \item \textbf{Jelaskan fungsi dari masing-masing library yang berhasil Anda install!}
    
    \textit{Berikut fungsi dari masing-masing library yang berhasil saya install:
    \begin{itemize}
        \item \textbf{Librosa:} Library untuk analisis dan pemrosesan audio, seperti ekstraksi fitur audio.
        \item \textbf{Soundfile:} Library untuk membaca dan menulis file audio.
        \item \textbf{Scipy:} Library untuk komputasi ilmiah, termasuk pemrosesan sinyal.
        \item \textbf{OpenCV:} Library untuk pemrosesan gambar dan video.
        \item \textbf{Pillow:} Library untuk manipulasi gambar.
        \item \textbf{Scikit-image:} Library untuk pemrosesan gambar berbasis NumPy.
        \item \textbf{Matplotlib:} Library untuk visualisasi data, termasuk plotting grafik.
        \item \textbf{MoviePy:} Library untuk pengeditan video.
        \item \textbf{Numpy:} Library untuk komputasi numerik dengan array multidimensi.
        \item \textbf{Pandas:} Library untuk manipulasi dan analisis data.
        \item \textbf{Jupyter:} Alat untuk membuat dan berbagi dokumen yang berisi kode, visualisasi, dan teks.
    \end{itemize}}
\end{enumerate}

\subsection{Troubleshooting}
\textbf{Dokumentasikan masalah yang Anda hadapi (jika ada) dan cara mengatasinya:}

\begin{itemize}
    \item \textbf{Masalah 1:} \textit{Ketika saya ingin menjalankan script test audio dan image, muncul error ModuleNotFoundError: No module named 'numpy'.}
    
    \textbf{Solusi:} \textit{Setelah saya cek, modul numpy sudah terinstall di environment uv saya. Ternyata masalahnya adalah saya lupa mengaktifkan environment uv sebelum menjalankan script. Setelah saya aktifkan environment uv, script dapat berjalan dengan lancar tanpa error.}
    
    % \item \textbf{Masalah 2:} \textit{[Deskripsi masalah]}
    
    % \textbf{Solusi:} \textit{[Cara mengatasi]}
\end{itemize}

\section{Export Environment untuk Reproduksi}
Sebagai langkah terakhir, export environment Anda agar dapat direproduksi:

% \subsection{Untuk Conda}
% \begin{lstlisting}[language=bash, caption=Export conda environment]
% conda env export > environment.yml
% \end{lstlisting}

% \subsection{Untuk venv/uv}
% \begin{lstlisting}[language=bash, caption=Export pip requirements]
% pip freeze > requirements.txt
% \end{lstlisting}

% \textbf{Copy-paste isi file environment.yml atau requirements.txt di sini:}

\begin{lstlisting}[caption=Environment/Requirements file]
audioop-lts==0.2.2
audioread==3.0.1
certifi==2025.8.3
cffi==1.17.1
charset-normalizer==3.4.3
colorama==0.4.6
contourpy==1.3.3
cycler==0.12.1
decorator==5.2.1
fonttools==4.59.2
idna==3.10
imageio==2.37.0
imageio-ffmpeg==0.6.0
joblib==1.5.2
kiwisolver==1.4.9
lazy_loader==0.4
librosa==0.11.0
llvmlite==0.44.0
matplotlib==3.10.5
moviepy==2.2.1
msgpack==1.1.1
networkx==3.5
numba==0.61.2
numpy==2.2.6
opencv-python==4.12.0.88
packaging==25.0
pillow==11.3.0
platformdirs==4.4.0
pooch==1.8.2
proglog==0.1.12
pycparser==2.22
pyparsing==3.2.3
python-dateutil==2.9.0.post0
python-dotenv==1.1.1
requests==2.32.5
scikit-image==0.25.2
scikit-learn==1.7.1
scipy==1.16.1
six==1.17.0
soundfile==0.13.1
soxr==0.5.0.post1
standard-aifc==3.13.0
standard-chunk==3.13.0
standard-sunau==3.13.0
threadpoolctl==3.6.0
tifffile==2025.8.28
tqdm==4.67.1
typing_extensions==4.15.0
urllib3==2.5.0
uv==0.8.13
\end{lstlisting}

\section{Kesimpulan}
\textbf{Tuliskan kesimpulan Anda mengenai:}
\begin{itemize}
    \item Pengalaman setup Python environment untuk multimedia
    \item Persiapan untuk project multimedia selanjutnya
    \item Saran untuk mahasiswa lain yang akan melakukan setup serupa
\end{itemize}

\textit{Pengalaman saya dalam melakukan setup Python environment untuk multimedia ini sangat menyenangkan dan memberikan banyak wawasan baru. Saya belajar tentang pentingnya manajemen environment untuk menghindari konflik antar library. Proses instalasi library-library multimedia juga berjalan lancar tanpa kendala yang besar, untuk persiapan project multimedia selanjutnya, saya merasa lebih siap karena sudah memahami cara mengatur environment dan menginstall library yang diperlukan. Saran saya untuk mahasiswa lain yang akan melakukan setup serupa adalah untuk selalu memastikan environment sudah aktif sebelum menjalankan script, serta mendokumentasikan setiap langkah yang dilakukan agar mudah untuk direproduksi di project selanjutnya.}

\section{Referensi}
Sertakan referensi yang Anda gunakan selama proses setup dan troubleshooting.

\newpage
\bibliographystyle{IEEEtran}
\bibliography{Referensi}
\nocite{chatgpt-conda-venv-uv}
\end{document}